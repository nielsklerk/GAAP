\chapter{Methods}


\section{Color Determination Algorithm}

Aperture photometry can be written like

\begin{equation}
    F_A = \int dxdyO(x,y)W_A(x,y)
\end{equation}

where $O(x,y)$ is the observed data and $W_A(x,y)$ some weight function. We know that $O(x,y)$ is the convolution of the intrinsic image with the PSF

\begin{equation}
    O(x,y) = \int dx'dy'P(x-x', y-y')I(x',y')=(P\otimes I)(x,y)
\end{equation}

We can use this to rewrite

\begin{equation}
    F_A = \int dxdy\int dx'dy'P(x-x', y-y')I(x',y')W_A(x,y)
\end{equation}

a change of coordinates, then gives

\begin{equation}
    F_A = \int dxdyI(x,y)\tilde{W}_A(x,y)
\end{equation}

where 

\begin{equation}
    \tilde{W}_A(x,y) = \int dx'dy'P(x'-x, y'-y)W_A(x',y')=(\bar{P}\otimes W_A)(x,y)
\end{equation}

where $\bar{P}(x,y)=P(-x,-y)$.

The algorithm then works as follows:
\begin{enumerate}
    \item Choose an aperture function $\tilde{W}_A$, which is going to be applied to all bands
    \item Calculate the post-seeing aperture $\tilde{W}_A^i$ by deconvolving the aperture function $\tilde{W}_A$ by the reflected PSF $\bar{P}_i$, of band $i$, $W^i_A=\tilde{W}_A\otimes^{-1}\bar{P}_i$
    \item Calculate the aperture flux for each band using the deconvolved aperture function $W_A^i$.
\end{enumerate}

The deconvolution can be calculated by examining the equations in the frequency domain 

\begin{equation}
    \mathcal{F}\{\tilde{W}_A\}=\mathcal{F}\{\bar{P}_i\}\cdot\mathcal{F}\{W_A^i\},
\end{equation}

which results in

\begin{equation}
    W^i_A = \mathcal{F}^{-1}\left\{\frac{\mathcal{F}\{\tilde{W}_A\}}{\mathcal{F}\{\bar{P}_i\}}\right\}.
\end{equation}
To improve the numerical stability of this function, a factor $K=10^{-15}$ is added

\begin{equation}
    W^i_A = \mathcal{F}^{-1}\left\{\frac{\mathcal{F}\{\bar{P}_i\}^*}{|\mathcal{F}\{\bar{P}_i\}|^2+K}\cdot\mathcal{F}\{\tilde{W}_A\}\right\}.
\end{equation}


If the weight matrix is the same across different sources, except for a shift. The aperture flux becomes 

\begin{equation}
    F_A(x',y') = \int dxdyO(x,y)W_A(x-x',y-y') = (O \otimes \bar{W}_A)(x',y'),
\end{equation}

with $\bar{W}_A(x,y)=W_A(-x,-y)$.

As the image has discrete pixels, the formula for the aperture flux becomes

\begin{equation}
    F_A=\sum_iO_{i}w_{i}
\end{equation}

where $O$ and $w$ are the flattened versions of the observation and the weight, respectively. The uncertainty in the flux can be calculated using

\begin{equation}
    \sigma^2 = \sum_{i,j}w_iw_jC_{ij}
\end{equation}

where $C_{ij}$ is the covariance matrix.

\subsection{Uncorrelated Noise}
Assuming that the pixel variations are independent and have variance $\sigma_{\mathrm{pixel}}^2$, the covariance matrix can be written like $C_{ij}=\sigma_{\mathrm{pixel}}^2\delta_{ij}$. This simplifies the formula for the uncertainty in the flux to

\begin{equation}\label{eq: uncorrelated noise}
    \sigma^2 = \sigma_{\mathrm{pixel}}^2\sum_iw_i^2.
\end{equation}

Assuming the noise is symmetric around zero, we consider only negative pixel values. This is to exclude any signals other than the noise. Let $O_i$ be the pixel values at index $i$ and $O_-$ be the set of indexes where $O_i<0$

\begin{equation}
    \sigma_{pixel}^2 = \frac{1}{|O_{-}|} \sum_{i \in O_{-}} O_i^2
\end{equation}

\subsection{Correlated Noise}

\begin{equation}
    \sigma^2 = \sum_{x,y}\sum_{x',y'}W_A(x,y)C(x,y;x',y')W_A(x',y')
\end{equation}

We assume that the covariance of pixels only depends on the distance between the pixels

\begin{equation}
    C(x,y;x',y') \approx C(x'-x,y'-y)=C(\Delta x, \Delta y)
\end{equation}

\begin{equation}
    \sigma^2 = \sum_{x,y}\sum_{\Delta x,\Delta y}W_A(x,y)C(\Delta x, \Delta y)W_A(x+\Delta x,y + \Delta y)
\end{equation}

Lastly, we assume that the $C(\Delta x, \Delta y) = 0 \forall|\Delta x|, |\Delta y| > L$

This reduces the number of terms in the 

\begin{equation}
    \sigma^2 = \sum_{\Delta x=-L}^{+L}\sum_{\Delta y=-L}^{+L}C(\Delta x, \Delta y)\sum_{x,y}W_A(x,y)W_A(x+\Delta x,y + \Delta y)
\end{equation}

The $C(\Delta x, \Delta y)$ can be calculated using the 
formula

\begin{equation}
    C(\Delta x, \Delta y) = \langle \tilde{I}(x,y)\tilde{I}(x+\Delta x,y+\Delta y)\rangle 
\end{equation}

where $\tilde{I}(x,y)=I(x,y) - \langle I(x,y)\rangle$ with $I$ being a pure noise image.

\begin{equation}
    C(\Delta x, \Delta y) = \frac{\int dx dy \tilde{I}(x,y)\tilde{I}(x+\Delta x,y+\Delta y)}{\int dx dy} 
\end{equation}

\begin{equation}
    C(\Delta x, \Delta y) = \frac{\left(\tilde{I}(x,y) \otimes\tilde{I}(-x,-y)\right)(\Delta x, \Delta y)}{\int dx dy} 
\end{equation}


\subsection{Poisson Noise}

For the Poisson noise, it is assumed that the pixels are uncorrelated. However, unlike the variance in \autoref{eq: uncorrelated noise}, this noise depends on the location in the image. A characteristic property of Poisson noise is that the variance is equal to the mean of the value. 

\begin{equation}
    \sigma_{poisson}^2=\sum_iw_i^2O_i
\end{equation}

This only works if the image is in counts. If this is not the case, the observation should first be converted to counts, then the variance can be calculated, and lastly, the variance can be converted back to the original units of the image.


\section{Simulation Framework}

We define a fiducial model. 
We simulate an image of 100 by 100 pixels. We chose an elliptical Gaussian for the galaxy's intrinsic image, with standard deviations of 5 pixels along the y-axis and 10 pixels along the x-axis. The sum of the pixel values of this intrinsic image of the galaxy is 1000. This intrinsic image is then convolved with a normalized Gaussian PSF with a standard deviation of 3 pixels. The fiducial weight function is a normalized Gaussian with a standard deviation of 10 pixels. To add noise to the image, we take Gaussian-distributed values with a variance of 0.01. This noise is uncorrelated, so for the next part we can use \autoref{eq: uncorrelated noise}

Next, each component of the fiducial model is varied individually to investigate the consequences of each part on the measured flux and the estimated error.  


\subsection{Flux Accuracy}

To test the accuracy of the GAAP method, the flux determined from the pre-seeing image with the pre-seeing weight function

\begin{equation}
    F_{True} = \int dxdy I(x,y) \tilde{W}_A(x,y)
\end{equation}

with the flux measured with the deconvolved weight and the observed image with noise. 

\begin{equation}
    F_{Measured} = \int dxdy O(x,y) W_A(x,y)
\end{equation}


\subsection{Error Precision}


\subsection{Aperture Shape}

The best aperture shape is the one that follows the intrinsic shape of the source. 


\section{Observational Data}


\subsection{PSF Construction}

chimney plot


\subsection{Noise Square}


\section{Photometric Redshift Estimation}