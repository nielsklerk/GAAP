\chapter{Methods}

\section{Color Determination Algorithm}
\citep{Kuijken_2008}, \cite{Refregier_2003_1}, \cite{Refregier_2003_2} \cite{Kuijken_2015}

% \begin{equation}
%     L_{lmn}(a,b,c) = \tfrac{1}{\sqrt{\pi}}\int dx\; e^{-x^2}H_l(ax)H_m(bx)H_n(cx)
% \end{equation}

% Using the explicit expression of the Hermite polynomial

% \begin{align}
%     H_n(ax)&=\sum_{m=0}^{\lfloor\frac{n}{2}\rfloor}\frac{n!(-1)^m}{m!(n-2m)!}(2ax)^{n-2m}\\&=\sum_{m=0}^{\lfloor\frac{n}{2}\rfloor}h_{m,n}(a)x^{n-2m}
% \end{align}

% with 

% \begin{equation}
%     h_{m,n}(a)=\frac{n!(-1)^m(2a)^{n-2m}}{m!(n-2m)!}
% \end{equation}

% we find

% \begin{align}
%     L_{lmn}(a,b,c) &= \tfrac{1}{\sqrt{\pi}}\int dx\; e^{-x^2}\sum_{i=0}^{\lfloor\frac{l}{2}\rfloor}h_{i,l}(a)x^{l-2i}\sum_{j=0}^{\lfloor\frac{m}{2}\rfloor}h_{j,m}(b)x^{m-2j}\sum_{k=0}^{\lfloor\frac{n}{2}\rfloor}h_{k,n}(c)x^{n-2k}\\
%      &= \tfrac{1}{\sqrt{\pi}}\int dx\; e^{-x^2}\sum_{i=0}^{\lfloor\frac{l}{2}\rfloor}\sum_{j=0}^{\lfloor\frac{m}{2}\rfloor}\sum_{k=0}^{\lfloor\frac{n}{2}\rfloor}h_{i,l}(a)h_{j,m}(b)h_{k,n}(c)x^{l+m+n-2(i+j+k)}\\
%      &= \tfrac{1}{\sqrt{\pi}}\sum_{i=0}^{\lfloor\frac{l}{2}\rfloor}\sum_{j=0}^{\lfloor\frac{m}{2}\rfloor}\sum_{k=0}^{\lfloor\frac{n}{2}\rfloor}h_{i,l}(a)h_{j,m}(b)h_{k,n}(c)\int dx\; e^{-x^2}x^{l+m+n-2(i+j+k)}
% \end{align}

% Using the fact that $L_{lmn}$ is only non-zero when $l+m+n$ is even, we can use

% \begin{equation}
%     \int dx\;x^{2n}e^{-x^2}=\sqrt{\pi}\frac{(2n)!}{n!}\left(\frac{1}{2}\right)^{2n}
% \end{equation}

% to evaluate the integral.

% % \begin{align}
% %     L_{lmn}(a,b,c) &= \sum_{i=0}^{\lfloor\frac{l}{2}\rfloor}\sum_{j=0}^{\lfloor\frac{m}{2}\rfloor}\sum_{k=0}^{\lfloor\frac{n}{2}\rfloor}h_{i,l}(a)h_{j,m}(b)h_{k,n}(c)\frac{(l+m+n-2(i+j+k))!}{((l+m+n)/2-(i+j+k)!}\left(\frac{1}{2}\right)^{l+m+n-2(i+j+k)}
% % \end{align}

% \begin{align}
% L_{lmn}(a,b,c) &= 
% \sum_{i=0}^{\lfloor\frac{l}{2}\rfloor}
% \sum_{j=0}^{\lfloor\frac{m}{2}\rfloor}
% \sum_{k=0}^{\lfloor\frac{n}{2}\rfloor}
% h_{i,l}(a)\, h_{j,m}(b)\, h_{k,n}(c) \\
% &\quad\;\, \times 
% \frac{(l+m+n-2(i+j+k))!}
%      { \left( \frac{l+m+n}{2} - (i+j+k) \right)! }
% \left( \frac{1}{2} \right)^{l+m+n-2(i+j+k)}
% \end{align}

Aperture photometry can be written like

\begin{equation}
    F_A = \int\int dxdyO(x,y)W_A(x,y)
\end{equation}

where $O(x,y)$ is the observed data and $W_A(x,y)$ some weight function. We know that $O(x,y)$ is the convolution of the intrinsic image with the PSF

\begin{equation}
    O(x,y) = \int\int dx'dy'P(x-x', y-y')I(x',y')=P\otimes I
\end{equation}

We can use this to rewrite

\begin{equation}
    F_A = \int\int dxdy\int\int dx'dy'P(x-x', y-y')I(x',y')W_A(x,y)
\end{equation}

a change of coordinates, then gives

\begin{equation}
    F_A = \int\int dxdyI(x,y)\tilde{W}_A^P(x,y)
\end{equation}

where 

\begin{equation}
    \tilde{W}_A^P(x,y) = \int\int dx'dy'P(x'-x, y'-y)W_A(x',y')=\bar{P}\otimes W_A
\end{equation}

where $\bar{P}(x,y)=P(-x,-y)$.

The algorithm then works as follows:
\begin{enumerate}
    \item Choose an aperture function $\tilde{W}_A$, which is going to be applied to all bands
    \item Calculate the post-seeing aperture $\tilde{W}_A^i$ by deconvolving the aperture function $\tilde{W}_A$ by the reflected PSF $\bar{P}_i$, of band $i$, $W^i_A=\tilde{W}_A\otimes^{-1}\bar{P}_i$
    \item Calculate the aperture flux for each band using the deconvolved aperture function $W_A^i$.
\end{enumerate}

As the image has discrete pixels, the formula for the aperture flux becomes
\begin{equation}
    F_A=O_{i}w_{i}
\end{equation}
where $O$ and $w$ are the flattened versions of the observation and the weight, respectively. The uncertainty in the flux can be calculated using
\begin{equation}
    \sigma^2 = w_iw_jC_{ij}
\end{equation}
where $C_{ij}$ is the covariance matrix. Assuming that the pixel variations are independent and have variance $\sigma_{\mathrm{pixel}}^2$, the covariance matrix can be written like $C_{ij}=\sigma_{\mathrm{pixel}}^2\delta_{ij}$. This simplifies the formula for the uncertainty in the flux to
\begin{equation}
    \sigma^2 = \sigma_{\mathrm{pixel}}^2w_iw_i.
\end{equation}
Assuming that the pixel values are normally distributed around zero $x_i \sim\mathcal{N}(0, \sigma_{\mathrm{pixel}}^2)$
Choosing the uninformative conjugate prior $p(\sigma_{\mathrm{pixel}}^2)\sim IG(\alpha, \beta)$ with $\alpha=\beta=0$. This results in the posterior distribution $p(\sigma_{\mathrm{pixel}}^2|\{x_i\})\propto p(\{x_i\}|\sigma_{\mathrm{pixel}}^2)p(\sigma_{\mathrm{pixel}}^2)\propto IG(\alpha + \frac{N}{2}, \beta+\frac{\sum x_i^2}{2})$. Using the fact that the mean of the inverse gamma distribution is given by $\frac{\beta}{\alpha-1}$, we find
$\sigma_{\mathrm{pixel}}^2=\frac{\sum x_i^2}{n+2}$.

\begin{equation}
    SNR  = \frac{\sum_iO_iw_i}{\sqrt{}}
\end{equation}

The deconvolution can be calculated by looking at the equations in the frequency domain 

\begin{equation}
    \mathcal{F}\{\tilde{W}_A^P(x,y)\}=\mathcal{F}\{\bar{P}(x,y)\}\cdot\mathcal{F}\{W_A(x,y)\}
\end{equation}

which gives us 

\begin{equation}
    W^i_A = \mathcal{F}^{-1}\left\{\frac{\mathcal{F}\{\tilde{W}_A^P(x,y)\}}{\mathcal{F}\{\bar{P}(x,y)\}}\right\}
\end{equation}

This is the weight matrix that is
\begin{equation}
    F_A = \int\int dxdyO(x,y)W_A(x,y)
\end{equation}
The weight matrix is the same for different sources, other than the fact that it is shifted 
\begin{equation}
    F_A(x',y') = \int\int dxdyO(x,y)W_A(x-x',y-y') = \int\int dxdyO(x,y)W_A(-(x'-x),-(y'-y') \newline = (O \otimes \bar{W_A})(x',y')
\end{equation}
\begin{equation}
    F_A(x',y')  = (O \otimes \bar{W}_A)(x',y')
\end{equation}
with $\bar{W}_A(x,y)=W_A(-x,-y)$

\subsection{Noise Estimation}
The variance on the flux is given by the following equation
\begin{equation}
    \sigma^2 = w_iw_jC_{ij}
\end{equation}
where $C_{ij}$ is the covariance matrix. This value is independent of the observed object so the variance on the flux is the same for all objects in the image. 
\subsection{Uncorrelated Noise}
If we assume there is no correlation between the pixels values, we find that $C_{ij}=\sigma_{pixel}^2\delta_{ij}$ where $\sigma_{pixel}$ is the variance the the pixel values. Assuming that the noise is symmetric around zero, we will only consider the negative pixel values. This is to exclude any signals other that the noise. Let $O_i$ be the pixel values at index $i$ and $O_-$ be the set of indexes where $O_i<0$
\begin{equation}
\sigma_{pixel}^2 = \frac{1}{|O_{-}|} \sum_{i \in O_{-}} O_i^2
\end{equation} 
This results in the uncorrelated variance being:
\begin{equation}\label{eq: uncorrelated noise}
    \sigma_{uncorrelated}^2=\sigma_{pixel}^2\sum_iw_i^2
\end{equation}
\subsection{Correlated Noise}
\begin{equation}
    \sigma^2 = \sum_{x,y}\sum_{x',y'}W_A(x,y)C(x,y;x',y')W_A(x',y')
\end{equation}
We assume that the covariance of pixels only depends on the distance between the pixels
\begin{equation}
    C(x,y;x',y') \approx C(x'-x,y'-y)=C(\Delta x, \Delta y)
\end{equation}
\begin{equation}
    \sigma^2 = \sum_{x,y}\sum_{\Delta x,\Delta y}W_A(x,y)C(\Delta x, \Delta y)W_A(x+\Delta x,y + \Delta y)
\end{equation}
Lastly, we assume that the $C(\Delta x, \Delta y) = 0 \forall|\Delta x|, |\Delta y| > L$ 
This reduces the number of terms in the 
\begin{equation}
    \sigma^2 = \sum_{\Delta x=-L}^{+L}\sum_{\Delta y=-L}^{+L}C(\Delta x, \Delta y)\sum_{x,y}W_A(x,y)W_A(x+\Delta x,y + \Delta y)
\end{equation}
The $C(\Delta x, \Delta y)$ can be calculated using the 
formula
\begin{equation}
    C(\Delta x, \Delta y) = \langle \tilde{I}(x,y)\tilde{I}(x+\Delta x,y+\Delta y)\rangle 
\end{equation}
where $\tilde{I}(x,y)=I(x,y) - \langle I(x,y)\rangle$ with $I$ being a pure noise image.
\begin{equation}
    C(\Delta x, \Delta y) = \frac{1}{HW} \int\int dx dy \tilde{I}(x,y)\tilde{I}(x+\Delta x,y+\Delta y)
\end{equation}
\begin{equation}
    C(\Delta x, \Delta y) = \frac{1}{HW} (\tilde{I}(x,y) \otimes\tilde{I}(-x,-y))(\Delta x, \Delta y)
\end{equation}
Note: when $\Delta x = \Delta y = 0$, the expression reduces back to the expression derived for the uncorrelated noise

\section{Simulation Framework}
We define a fiducial model. 
We simulate an image of 100 by 100 pixels. We chose an elliptical Gaussian for the galaxy's intrinsic image, with standard deviations of 5 pixels along the y-axis and 10 pixels along the x-axis. This intrinsic image is then convolved with a Gaussian PSF with a standard deviation of 3 pixels. The fiducial weight function is a normalized Gaussian with a standard deviation of 8 pixels. To add noise to the image, we take Gaussian distributed values with a variance of \textbf{NOISE VARIANCE}. This noise is uncorrelated, so for the next part we can use \autoref{eq: uncorrelated noise}

\subsection{Color Precision: PSF Size}

\subsection{Color Precision: Aperture Size}

\subsection{Error Precision: PSF Size}
The error precision is independent of the size of the PSF. 
\subsection{Error Precision: Pixel Covariance}
The error precision depends both on the covariance of the pixel values and the distance of pixel values that is taken into the local covariance matrix

\subsection{Aperture Shape}
The best aperture shape is the one that follows the shape of the source. 

\section{Observational Data}

\section{Photometric Redshift Estimation}